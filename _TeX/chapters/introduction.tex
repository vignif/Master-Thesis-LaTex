%\chapter*{Introduction}
%The handshake event, commonly used, is a natural human interaction and is extensively used worldwide in events like: greetings, introduction routine between human beings and agreements. 
%The scientific approach to handshake between a human and a robot, therefore, must intrinsically deal with low levels human interactions and some assumption must be done in order to focus on the task.
%The handshake event can be divided in two parts: the approaching and the handshaking. This work is focused on the haptic sense involved during the handshake, knowing that the approaching is mainly executed using non haptic senses (f.i. vision).\\ %The full routine of the handshake event can be summarized in a starting time, a duration. Assuming the human to be in a leading position both, the starting point and the duration of the handshake are decided by the human.
%An interesting parameter involved in the handshake is the consensus; it allow the human to evaluate a handshake mixing aspects like: duration of the event, dynamics, force exchanged.
%An important part of this work is to test different controllers with the purpose of evaluating the consensus using closed loop controllers. 
%  
%Many research teams all over the world are focused on the human-robot physical interaction, this opens the topic to an interesting scientific discussion.  
\chapter*{Introduction}
Develop a robot capable of performing a smooth human-like handshake is still a highly interested topic in the scientific literature.
A natural handshake between two humans is a very complex task to replicate, this work just focuses on the interaction force between a robot hand and a human hand.
In many parts of the world, the handshake is an important interaction task both for business and social context \cite{chaplin2000handshaking}, and an important behaviour to identify is the consensus in the event. It is reasonable to assume that in human-human handshake it is reached using not just the haptic sense for the task. Due to the nature of this behaviour it is complex to embed inside a robot, participants will easily distinguish the event with respect to another human or to a robot. A human will naturally take into account for executing a handshake not just the grip force but the skin feedbacks like: temperature, humidity; vision and prior expectation are assumed to affect the task. However, there is little work in the literature studying human-human handshaking, and as such it is not yet possible to describe what constitutes a ‘good’ or a ‘bad’ handshake, or even describe a human-human handshake, in a quantitative manner.
In Human Robot Interaction \cite{sheridan2016human}, the handshake is a really interesting task to focus on, typically leader and follower roles are clearly defined, master action is measured and elaborated to generate reference inputs for the slave controller. In handshake this prior allocation of roles is not defined, it is an inherently bidirectional action in which both sides actively contribute to the task by applying an active and a reactive action at the same time.
The aspect taken in consideration in this work is the grasping force exchanged in the handshake therefore, an accurate choice has been done on the hardware to use in this work.
%These are some characteristics that are still not embedded into the hardware available in the market.
%The grasping force exchanged in the human-robot handshake event is a complex value to identify, this work is estimating the grasping force from values which can be clearly identified. 
