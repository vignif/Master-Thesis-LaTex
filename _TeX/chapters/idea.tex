\chapter{The Idea}
The idea is to create a closed loop controller for the human-robot handshake event, using an under actuated robot hand developed for research purposes and upgrading it with two independent FSR sensors which uses an Arduino uno in order to communicate the data.
The FSR sensors are located on the robotic hand so there are no wearing device on the human hand during the execution of the experiments.
This choice leads the work to be focused on the theoretical part of the handshake event, and potentially reach robust results. We are focusing on a general human-robot handshake, knowing that the interaction can vary with participants, f.i. the participant's hand size is affecting the firsts contact points or nominal strength to apply in the handshake can be affected by prior expectations. It is more meaningful then, to study individual differences once the generic case has been studied.
The hypothesis is that in human hand-shaking force control there is a balance between an intrinsic (open–loop) and extrinsic (closed–loop) contribution.
Is reasonable to assume that a robot handshake can be evaluated positively by a human if it is as similar as possible to a human handshake, for this reason the presented controllers target aim to mirror the behaviour in human-human handshake.
The robot hand is a soft underactuated anthropomorphic robot hand, the Pisa/IIT SoftHand, instrumented with pressure sensors in order to measure the grasping force exerted onto it, the robot grasping force is estimate from its pose.
The chosen robotic hand (Pisa/IIT SoftHand) has 19 degrees of freedom and its main characteristic is a single dc motor that is pulling a tendon which is embedded in each finger. This physical approach results in an under actuated robotic hand which can be controlled only by the dc motor (1 DOF) and can easily adapt to different configurations without modifying the reference position. 