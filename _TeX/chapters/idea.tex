\chapter{The Idea}
The idea is to create a closed loop controller for the human-robot handshake event, using a robot hand developed for research purposes and instrumenting it with two independent FSR sensors which uses an Arduino uno in order to communicate the data.
The FSR sensors are located on the robotic hand so there are no wearing device on the human hand during the execution of the experiments.
This choice leads the work to be focused on the theory of the handshake event, and potentially reach robust results. We are focusing on a general human-robot handshake, knowing that the interaction can vary with participants, f.i. the participant's hand size is affecting the firsts contact points or nominal strength to apply in the handshake can be affected by prior expectations. It is more meaningful then, to study individual differences once the generic case has been studied.
The hypothesis is that in human hand-shaking force control there is a balance between an intrinsic (open–loop) and extrinsic (closed–loop) contribution.
It is reasonable to assume that a robot handshake can be evaluated positively by a human if it is as similar as possible to a human handshake, for this reason the presented controllers aim to mirror the behaviour in human-human handshake.
The robot hand is a soft under actuated anthropomorphic robot hand, the Pisa/IIT SoftHand exploiting the idea of synergies \cite{catalano2014adaptive}. The hardware is instrumented with pressure sensors in order to measure the grasping force exerted onto it.% the robot grasping force is estimate from its pose.
The chosen robotic hand (Pisa/IIT SoftHand) has a single actuated degree of freedom and embeds a dc motor pulling a tendon through in each finger. This physical approach results in a robot hand which can easily adapt to different configurations without modifying the reference position. 
%A really first approach to human-robot handshake is to read the task as a leader/follower task. The first controller according to this, is trying to follow the human grasping force in the event (closed loop). The robot is reacts to the human force, an opposite approach is to verity whether the human in able to follow the robot force. The robot in this example is initially triggered by the human and then follows its own behaviour 

\todo{insert figure showing the variation from closed loop to open loop}